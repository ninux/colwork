\section{Einleitung}

\subsection{Person}
\begin{frame}
	\frametitle{Wer bin ich? \hfill{} \footnotesize{LuXeria}}
	\begin{block}{Personalien, Ausbildung \& Hobby}
		\begin{itemize}
			\item Ervin Mazlagic
			\item Perlen (LU), Biha\'c (BiH)
			\item Student ET, HSLU-T\&A
			\item Elektroniker - R\&D
			\item Schindler Aufzüge AG, FU-HW-Entwicklung
			\item President des LuXeria
		\end{itemize}
	\end{block}
\end{frame}

\subsection{Ziele}
\begin{frame}
	\frametitle{Was soll erreicht werden? \hfill{} \footnotesize{LuXeria}}
	\begin{block}{Nach dieser Präsentation können Sie}
		\begin{itemize}
			\item WYSIWYG von \TeX~\& \LaTeX~unterscheiden
			\item Git und seine Eigenschaften nennen
			\item Funktionen und Philisophie von Github nennen
			\item Einsatz von \LaTeX, Git und Github abschätzen
			\item selbstständig weitere Informationen finden
		\end{itemize}
	\end{block}
\end{frame}

\subsection{Zielgruppen}
\begin{frame}
	\frametitle{An wen richtet sich diese Präsentation? \hfill{} \footnotesize{LuXeria}}
	\begin{exampleblock}{An jene die}
		\begin{itemize}
			\item Texte aller Art zusammen erstellen/bearbeiten wollen
			\item Texte mit Projektmanagement verwalten möchten
			\item \LaTeX, Git und Github nicht kennen/gewohnt sind
			\item eine Alternative zu kollaborativem Texten suchen
		\end{itemize}
	\end{exampleblock}

	\begin{alertblock}{Nicht an jene die}
		\begin{itemize}
			\item Git und Github bereits kollaborativ benutzen
			\item auf MS-Office-Formate nicht verzichten können
		\end{itemize}
	\end{alertblock}
\end{frame}

\subsection{Programm}
\begin{frame}
\frametitle{Programmübersicht \hfill{} \footnotesize{LuXeria}}
	\tableofcontents[hideallsubsections]
\end{frame}
