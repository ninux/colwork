\section{\TeX~\& \LaTeX}
\begin{frame}
    \frametitle{}
    \tableofcontents[currentsection,hideothersubsections]
\end{frame}

\subsection{WYSIWYG und Andere}
\begin{frame}
	\frametitle{WYSIWYG \hfill{} \footnotesize{LuXeria}}
    \framesubtitle{What You See Is (More Or Less) What You Get!}
    \begin{columns}
        \begin{column}{5cm}
            \begin{alertblock}{WYSI(MOL)WYG}
                \begin{itemize}
                    \item Was du siehst ist das, was du bekommst!
                    \item Interagiert mit Systemdaten (Druckertreiber etc.)
                    \item unvorhersagbar
                    \item undurchsichtig
                    \item Bsp. MS-Office
                \end{itemize}
            \end{alertblock}
        \end{column}
        \begin{column}{5cm}
            \begin{exampleblock}{\TeX~\& \LaTeX}
                \begin{itemize}
                    \item Was du bekommst ist das, wonach du gefragt hast!
                    \item System- und Plattformunabhängig
                    \item vorhersagbar (Programm)
                    \item Frei (free speech \& free beer)
                    \item professioneller Satz
                \end{itemize}
            \end{exampleblock}
        \end{column}
    \end{columns}
\end{frame}

\begin{frame}
	\frametitle{WYSIWYG \hfill{} \footnotesize{LuXeria}}
	\framesubtitle{Warum ist WYSIWYG problematisch?}
	\begin{alertblock}{Eigenschaften vieler WYSIWYG-Programme}
		\begin{itemize}
			\item isoliert Autoren
			\item Austausch ist erschwert/behindert
			\item viele Abhängigkeiten (SW-Versionen, HW, System ...)
			\item (hohe Kosten)
			\item unerklärliches Verhalten
			\item erzeugt überladene Dateien ($\Rightarrow$ Crash)
		\end{itemize}
	\end{alertblock}
\end{frame}

\subsection{\LaTeX}
\begin{frame}[fragile]
    \frametitle{\LaTeX \hfill{} \footnotesize{LuXeria}}
    \framesubtitle{Aufbau von \TeX-Dokumenten}
    \begin{columns}
        \begin{column}{5cm}
            \begin{block}{Dokument}
                \begin{columns}
                \begin{column}{4cm}
                \begin{exampleblock}{Präambel}
                    Layout, Definitionen, Bedingungen usw.
                \end{exampleblock}
                \begin{exampleblock}{Dokumenteninhalt}
                    Reiner Inhalt
                    \begin{columns}
                        \begin{column}{3cm}
                            \begin{alertblock}{externe Inhalte}
                                Verzeichnisse, Bilder usw.
                            \end{alertblock}
                        \end{column}
                    \end{columns}
                \end{exampleblock}
            \end{column}
            \end{columns}
            \end{block}
            %\end{column}
        \end{column}
        \begin{column}{5cm}
            \begin{block}{Beispiel}
            \begin{lstlisting}
\documentclass{beamer}
...
\usepackage[utf8]{inputenc}
\usepackage[ngerman]{babel}
...
\begin{document}
  \tableofcontents
            
  \section{Einleitung}
    \section{Einleitung}

\subsection{Person}
\begin{frame}
	\frametitle{Wer bin ich? \hfill{} \footnotesize{LuXeria}}
	\begin{block}{Personalien, Ausbildung \& Hobby}
		\begin{itemize}
			\item Ervin Mazlagic
			\item Perlen (LU), Biha\'c (BiH)
			\item Student ET, HSLU-T\&A
			\item Elektroniker - R\&D
			\item Schindler Aufzüge AG, FU-HW-Entwicklung
			\item Präsident des LuXeria
		\end{itemize}
	\end{block}
\end{frame}

\subsection{Ziele}
\begin{frame}
	\frametitle{Was soll erreicht werden? \hfill{} \footnotesize{LuXeria}}
	\begin{block}{Nach dieser Präsentation können Sie}
		\begin{itemize}
			\item WYSIWYG von \TeX~\& \LaTeX~unterscheiden
			\item Git und seine Eigenschaften nennen
			\item Funktionen und Philisophie von Github nennen
			\item Einsatz von \LaTeX, Git und Github abschätzen
			\item selbstständig weitere Informationen finden
		\end{itemize}
	\end{block}
\end{frame}

\subsection{Zielgruppen}
\begin{frame}
	\frametitle{An wen richtet sich diese Präsentation? \hfill{} \footnotesize{LuXeria}}
	\begin{exampleblock}{An jene die}
		\begin{itemize}
			\item Texte aller Art zusammen erstellen/bearbeiten wollen
			\item Texte mit Projektmanagement verwalten möchten
			\item \LaTeX, Git und Github nicht kennen/gewohnt sind
			\item eine Alternative zu kollaborativem Texten suchen
		\end{itemize}
	\end{exampleblock}

	\begin{alertblock}{Nicht an jene die}
		\begin{itemize}
			\item Git und Github bereits kollaborativ benutzen
			\item auf MS-Office-Formate nicht verzichten können
		\end{itemize}
	\end{alertblock}
\end{frame}

\subsection{Programm}
\begin{frame}
\frametitle{Programmübersicht \hfill{} \footnotesize{LuXeria}}
	\tableofcontents[hideallsubsections]
\end{frame}

    
\end{document}          
            \end{lstlisting}
            \end{block}
        \end{column}
    \end{columns}
\end{frame}


\begin{frame}[fragile]
    \frametitle{\LaTeX \hfill{} \footnotesize{LuXeria}}
    \framesubtitle{Immer \& überall das selbe Ergebnis!}
    \begin{columns}
        \begin{column}{5cm}
            \begin{block}{\LaTeX~Code}
                
\begin{lstlisting}{Bsp}
Einige Vorteile von \LaTeX:\\
% Hier folgt 
% eine Aufzaehlung
\begin{itemize}
  \item Freie Software
  \item $\sum\frac{n}{1-n^2}$
  \item Professioneller Satz
  \item Plattformunabhängig
  \item Logisch/ftrukturiert
  \item Einheitlich
  \item Einfache Verwaltung
\end{itemize}
\end{lstlisting}
                
            \end{block}
        \end{column}
        \begin{column}{5cm}
            \begin{block}{Ausgabe}
                Einige Vorteile von \LaTeX:\\
                % eine Aufzaehlung
                \begin{itemize}
                    \item Freie Software
                    \item $\sum\frac{n}{1-n^2}$
                    \item Professioneller Satz
                    \item Plattformunabhängig
                    \item Logisch/Strukturiert
                    \item Einheitlich
                    \item Einfache Verwaltung
                \end{itemize}
            \end{block}
        \end{column}
    \end{columns}
\end{frame}

\begin{frame}[fragile]
    \frametitle{\LaTeX~--- Environments \hfill{} \footnotesize{LuXeria}}
    \framesubtitle{Mathematik -- Inline \& abgesetzt}
    \begin{columns}
        \begin{column}{5cm}
            \begin{block}{Mathematik im Satz}
\begin{lstlisting}
... dann ist
$ nev\pi \frac{d^2}{4} $ 
die Ladungsmenge, die pro ...
\end{lstlisting}
            \end{block}
            \begin{block}{Mathematik abgesetzt}
\begin{lstlisting}
... berechnet sich nach Gauss: 
\[ \oint E' \cdot dA' = 
   \frac{1}{\epsilon_0} 
   \int \rho' dV' 
\]
\end{lstlisting}
            \end{block}
        \end{column}
        \begin{column}{5cm}
            \begin{block}{Mathematik im Satz}
            ...dann ist $nev\pi \frac{d^2}{4}$ die Ladungsmenge,
            die pro ...% Zeiteinheit durch den ganzen Leiterquerschnitt fliesst,
            %also die Stromstärke $I$.
            \end{block}
            \begin{block}{Mathematik abgesetzt}
            ... berechnet sich nach Gauss:
            \[ \oint E' \cdot dA' = \frac{1}{\epsilon_0} \int \rho' dV' \]
            \end{block}
        \end{column}
    \end{columns}
\end{frame}

\begin{frame}[fragile]
    \frametitle{\LaTeX~--- Environments \hfill{} \footnotesize{LuXeria}}
    \framesubtitle{TikZ -- TikZ ist kein Zeichenprogramm}
    \begin{columns}
        \begin{column}{6cm}
            \begin{block}{Pie-Chart Code}
\begin{lstlisting}
\begin{figure}
  \begin{tikzpicture}
    \pie[radius=1.75]{
        15/ HSLU,
        35/ ETH,
        10/ UZH,
        25/ MIT,
        15/ k.A}
  \end{tikzpicture}
  \caption{Ein 'pie'-Chart}
\end{figure}
\end{lstlisting}
            \end{block}
        \end{column}
        \begin{column}{5cm}
            \begin{figure}
                \begin{tikzpicture}
                    \pie[radius=1.75]{
                        15/ HSLU,
                        35/ ETH,
                        10/ UZH,
                        25/ MIT,
                        15/ k.A}
                \end{tikzpicture}
                \caption{Ein 'pie'-Chart}
            \end{figure}
        \end{column}
    \end{columns}
\end{frame}

\begin{frame}[fragile]
    \frametitle{\LaTeX~--- Environments \hfill{} \footnotesize{LuXeria}}
    \framesubtitle{TikZ -- TikZ ist kein Zeichenprogramm}
    \begin{columns}
        \begin{column}{5cm}
            \begin{block}{Bar-Chart Code}
                \begin{lstlisting}
\begin{figure}
  \begin{bchart}[step=20, 
                 max=60,
                 scale=0.65]
     \bcbar[text=HSLU]{15}
     \bcbar[text=ETH]{35}
     \bcbar[text=UZH]{10}
     \bcbar[text=MIT]{25}
     \bcbar[text=k.A]{15}
   \end{bchart}
   \caption{Ein 'bar'-Chart}
 \end{figure}
                \end{lstlisting}
            \end{block}
        \end{column}
        
        \begin{column}{6cm}
            \begin{figure}
                \begin{bchart}[step=10, max=40, scale=0.65]
                    \bcbar[text=HSLU]{15}
                    \bcbar[text=ETH]{35}
                    \bcbar[text=UZH]{10}
                    \bcbar[text=MIT]{25}
                    \bcbar[text=k.A]{15}
                \end{bchart}
                \caption{Ein 'bar'-Chart}
            \end{figure}
        \end{column}
    \end{columns}
\end{frame}

\begin{frame}[fragile]
    \frametitle{\LaTeX~--- Environments\hfill{} \footnotesize{LuXeria}}
    \framesubtitle{Circuitikz -- Analogtechnik}
    \begin{columns}
        \begin{column}{6cm}
            \begin{block}{Zweipol-Schaltung}
\begin{lstlisting}
\ctikzset{bipoles/length=1cm}
\begin{circuitikz}[scale=0.9]\draw
  (0,0) node[anchor=east]{B}
  to[short, o-*](1,0)
  to[R=$R_1$, *-*](1,2)
  to[R=$R_2$, *-*](3,2)--(4,2)
  to[V=$U_q$](4,0)--(3,0)
  to[R=$R_3$, *-*](3,2)
  (3,0)--(1,0)
  (1,2) to[short,-o]
  (0,2) node[anchor=east]{A}
;\end{circuitikz}
\end{lstlisting}
            \end{block}
        \end{column}
        \begin{column}{5cm}
            \ctikzset{bipoles/length=1cm}
            \begin{circuitikz}[scale=0.9]\draw
                (0,0) node[anchor=east]{B}
                to[short, o-*](1,0)
                to[R=$R_1$, *-*](1,2)
                to[R=$R_2$, *-*](3,2)--(4,2)
                to[V=$U_q$](4,0)--(3,0)
                to[R=$R_3$, *-*](3,2)
                (3,0)--(1,0)
                (1,2) to[short,-o](0,2) node[anchor=east]{A}
            ;\end{circuitikz}
        \end{column}
    \end{columns}
\end{frame}

\begin{frame}[fragile]
    \frametitle{\LaTeX~--- Environments \hfill{} \footnotesize{LuXeria}}
    \framesubtitle{Circuitikz -- Digitaltechnik}
    \begin{columns}
        \begin{column}{6cm}
            \begin{block}{DNF-Schaltung}
\begin{lstlisting}
\begin{circuitikz}\draw
  (0,2) node[and port] (and1) {}
  (0,0) node[and port] (and2) {}
  (2,1) node[xnor port] (xnor1) {}
  (and1.out)-|(xnor1.in 1)
  (and2.out)-|(xnor1.in 2)
  % Beschriftung
  (and1.in 1) node[anchor=east] {a}
  (and1.in 2) node[anchor=east] {b}
  ...
;\end{circuitikz}
\end{lstlisting}
            \end{block}
        \end{column}
        \begin{column}{5cm}
           \begin{circuitikz}\draw
        (0,2) node[and port] (and1) {}
        (0,0) node[and port] (and2) {}
        (2,1) node[xnor port] (xnor) {}
        (and1.out)-|(xnor.in 1)
        (and2.out)-|(xnor.in 2)
        (and1.in 1) node[anchor=east] {a}
        (and1.in 2) node[anchor=east] {b}
        (and2.in 1) node[anchor=east] {c}
        (and2.in 2) node[anchor=east] {d}
        (xnor.out) node[anchor=west] {y}
       ;\end{circuitikz}
        \end{column}
    \end{columns}
\end{frame}
